% $Log: abstract.tex,v $
% Revision 1.1  93/05/14  14:56:25  starflt
% Initial revision
%
% Revision 1.1  90/05/04  10:41:01  lwvanels
% Initial revision
%
%
%% The text of your abstract and nothing else (other than comments) goes here.
%% It will be single-spaced and the rest of the text that is supposed to go on
%% the abstract page will be generated by the abstractpage environment.  This
%% file should be \input (not \include 'd) from cover.tex.
This thesis analyzes and discusses the effectiveness of social efforts to achieve collective action amongst Internet network operators in order to manage the growth of the Internet routing table. The size and rate of growth of the Internet routing table is an acknowledged challenge impeding the scalability of our BGP interdomain routing architecture. While most of the work towards a solution to this problem has focused on architectural improvements, an effort launched in the 1990s called the CIDR Report attempts to incentivize route aggregation using social forces and norms in the Internet operator community. This thesis analyzes the behavior of Internet network operators in response to the CIDR Report from 1997 through 2010 to determine whether the Report was effective in achieving this goal.

While it is difficult to causally attribute aggregation behavior to appearance on the CIDR report, there is a clear trend for networks to improve their prefix aggregation following an appearance on the CIDR Report compared to untreated networks. This suggests that the CIDR Report was indeed likely effective, although the routing table continued to grow. This aggregation improvement is most prevalent early in the study period and becomes less apparent as time goes on. Potential causes of the apparent change in efficacy of the Report are discussed and examined using Ostrom's Common Pool Resource framework. The thesis then concludes with a discussion of options for mitigating routing table growth, including the continued use of community forces to better manage the Internet routing table.