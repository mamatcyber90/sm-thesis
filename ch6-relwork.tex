\chapter{Related Work}
\label{chap:relwork}

The important central role played by interdomain routing in the Internet, along
with the problems that many have seen emerge in this space over time, has
caused this topic to be a focus of much work on the part of academic
researchers, network equipment vendors, and network operators. This work can be
classified as that which is specifically focused on the problem of growth of
the routing table due to network operations, as well as work more broadly
focused on the problems of our current interdomain routing architecture. Both
of these topics are discussed in more detail in following sections. A section
on the limited literature and other information about the Internet operations
community and its norms, ethos, and characteristics is also included because of
the importance of this community in producing the social forces that could make
the CIDR Report effective.

Note that sources from the Internet operations community, while perhaps less
academically credible because they are typically not published in traditional
peer-reviewed forums, are included here and elsewhere in this thesis because
they are extremely valuable for gaining a more thorough and comprehensive
understanding of this topic.

% major hole -- methodological literature

\section{Analyses of routing table growth \& deaggregation}

Some of the work in this space has been devoted specifically to the topic of
the Internet routing table, arguably because it's both the immediate source of
scaling problems and because it's something that researchers, vendors, and
operators all think about. This work has generally focused on characterizing
the growth of the routing table over time and attempting to understand the
sources of or reasons for the growth. These works often mention the CIDR Report
as a source of information but do not discuss its efficacy.

Beyond some of the ad-hoc analysis that took place during the supernetting
\cite{rfc1338} and ultimately CIDR \cite{rfc1519} design and deployment,
\cite{Huston:2001bs} is one of the first works analyzing the Internet routing
table. In this paper, Huston analyzes the size and growth of the BGP routing
table from 1988-2001 and identifies four phases of growth surrounding the
deployment of CIDR. He finds that while CIDR deployment did reduce the size of
the routing table and limit the exponential growth preceding its deployment to
linear growth shortly after, growth did ultimately return to exponential again
as the Internet grew in commercial importance starting in the late 1990s. While
not specifically determining the contributions of various factors to the
observed growth, he does note that traffic engineering and multihoming are
increasing over time. Huston concludes by casting the problem as a ``tragedy of
the commons'', lamenting that the usual solution of a central regulator is not
obviously available to a globally distributed system like the Internet.
Interestingly, or perhaps tellingly, Huston does not acknowledge the CIDR
Report as a regulatory mechanism for this role.

Bu et al. \cite{Bu:2004fk} go on to analyze the routing table with a focus on
determining the contributions that a number of operational behaviors have on
routing table growth: load balancing (traffic engineering), multi-homing, RIR
address fragmentation, and failure to aggregate. Like this thesis, they also
use Route Views as their data source, verifying that it provides a sufficiently
comprehensive view of the Internet routing table. They perform an analysis of
aggregation potential similar to both the CIDR Report and the analysis of this
thesis, though they perform aggregation solely based on routing policy, rather
than prefix hierarchy as well. The authors of this paper conclude that address
fragmentation caused by RIR allocation policy was the greatest source of
prefixes in the routing table as of 2002, with multi-homing, traffic
engineering and failure to aggregate contributing 30\%, 25\%, and 20\%
additional prefixes respectively. They also find that traffic engineering was
the fastest-growing cause of routing table growth.

A more recent study by Cittadini et al. \cite{Cittadini:2010pi} provides an
update about the state of routing table deaggregation and update dynamics, and
also dispels some myths about where the blame lies for deaggregation in the
DFZ. This is a useful and welcome update given changes in the Internet since
the previous works. In spite of the Internet's evolution, they find that
neither the fraction of the routing table that is deaggregated nor the fraction
of ASes performing traffic engineering has changed noticeably since 2001, but
instead has remained proportional to the size of the routing table. They also
go on to analyze the distribution of deaggregation in different populations
(i.e. ISPs, enterprises, etc.) and find that while a disproportionately small
number of ASes advertise most of the routes in the DFZ, the proportion of these
``bad guys'' has not changed over time. These conclusions are interesting and
present a more reasonable perspective on the data analyzed by Bu and Huston in
that they consider the results relative to the routing table.
%The observation that the small proportion of ``bad guys'' has remained
%suggests that community forces might still possibly be valuable.

In good contrast to the more academically-oriented work in this space,
\cite{Zhao:2001ly} is written by a trio of experienced network operators and
describes the operational concerns regarding routing table growth, as well as
the challenges in implementing any proposed solution in a production service
provider environment, such as the business case for upgrades and the typical
5-7 year depreciation cycle for routers and other equipment. In
\cite{Steenbergen:2010nx}, Steenbergen presents another operator's perspective,
taking up a number of similar hypothesis regarding the cause of table growth as
in \cite{Bu:2004fk} and \cite{Cittadini:2010pi}, illustrating how these factors
are indeed present in the table, and discussing operational solutions to
minimize these problems. Finally, \cite{Smith:2006vn} presents the conclusions
of a more qualitative study by a European network operators group, which did
mention the CIDR Report and the ``CIDR Police'' (discussed below) as mechanisms
to mitigate routing table growth.

%Some other analyses of note include a a small study of deaggregation behavior
%in \cite{Gagliano:2009zr} where, like this thesis, the aggregation behavior is
%analyzed at the granularity of ASes rather than the routing table as a whole.
%Game theoretic modelling: \cite{Kalogiros:2009ly}

\section{Improving Interdomain Routing Scalability}

Much more of the work in this space, mainly by academics and researchers, has
been focused on the broader problems with our current architecture for
interdomain routing. Along with problem statements, a number of technical
solutions have been proposed, both incremental and radical (involving full
protocol changes or clean-slate redesigns), but most have yet to gain any
traction within vendor and operator communities. A few non-technical solutions
based on economic incentives or operational practices have been proposed or
attempted to resolve the more specific routing table growth problem. These
appear to be far less popular than technical solutions to the problem, and
while these ideas are often articulated in operator forums, they also have not
gained any traction.

At high level, \cite{Yannuzzi:2005hc} captures a number of challenges and
proposed improvements to interdomain routing within the context of arbitrary AS
topologies and a BGP-like routing algorithm. \cite{Handley:2006kx}, in the
context of describing a number of broader issues facing the Internet
architecture, discusses the scaling and reliability challenges of interdomain
routing and the even larger challenges of deploying a replacement until the
incentives are sufficient to do so. In \cite{Feamster:2004nx}, Feamster et al.
discuss a number of specific problems in BGP introduced by the design goals of
policy routing and scalability, with the intent to direct research towards
designing more robust interdomain routing protocols. The IETF and IAB have also
studied the problem of routing scalability more broadly in \cite{rfc4984}, with
the understanding that some of the scaling issues may come from our current
architecture and instead produced broader problem statements and criteria that
proposed solutions to the routing scalability problem must meet.

A number of small proposals to incrementally alter BGP-speaking routers have
emerged, such as routers that discard information and retrieve it later if
necessary \cite{Karpilovsky:2006ys}, and routers that aggregate their FIB
internally, similar to the aggregation performed by the CIDR Report, to
conserve memory without altering routing policy \cite{Zhao:2010vn}. Operational
techniques are also available to operators, including route filtering based on
RIR policies to limit more specific routes used for traffic engineering, etc.
\cite{Bellovin:2001qf}, although this technique is likely not to be effective
any more as RIRs have continued reducing their minimum allocation size with the
exhaustion of IPv4 addresses. Finally, routers can also be configured to
perform virtual aggregation, such as in \cite{Ballani:2009tg}, where a routing
table is effectively spread across multiple routers, thus performing full
lookups in two or more hops instead of one. All of these options are attractive
for individual providers as immediate benefit in terms of routing table
reduction can be realized without requiring coordination with others or a
protocol change, and are likely to be easily implemented with a software or
configuration update. However, while all of these approaches reduce the
routing table size and growth rate pragmatically, they do not alter BGP's
fundamental scaling characteristics, particularly with regard to update
messages.

In terms of architecture-level proposals, many solutions to the problem propose
to overcome the overloaded use of IP addresses as both topological locators and
endpoint identifiers. It is argued that eliminating this overload would allow
traffic engineering, multi-homing, and other behaviors to still occur while
reducing the size of routing table, as aggressive hierarchical aggregation
would be possible \cite{Quoitin:2007uq}. There have been a number of proposals
in this space, but one of the more well known efforts oriented at real
deployment is the Locator/ID Separation Protocol (LISP) \cite{lisp-id}, which
maintains two separate (IP) address spaces for identifiers and locators,
defines mappings between them for interdomain routing (i.e. which locator is
the destination for a given identifier), and encapsulates packets with this
appropriate mapping for interdomain routing. A more recent alternative that
appears more favorable for incremental deployment is the Identifier-Locator
Network Protocol (ILNP) \cite{Atkinson:2010zr}, utilizes the current IPv6
routing architecture along with DNS to provide locator-identity separation.
More radical proposals such as HLP \cite{Subramanian:2005ve} and NIRA
\cite{Yang:2007cr} are not incrementally deployable but offer new architectures
and models that are free from the warts of BGP and its implementation and
operation.

In contrast to CIDR and many of these other proposed schemes, including
locator-identifier separation, that rely on the aggregation properties of
hierarchical routing, \cite{Krioukov:2007fk} claim that we need a fundamentally
different approach to routing on the Internet as any hierarchical routing
algorithm will not scale well when applied to scale-free ``Internet-like''
topologies. They argue instead for \emph{compact routing} that by design scales
sub-linearly in the worst case, at the cost of selecting paths longer than the
shortest path between any two nodes. They make their argument from a mainly
theoretical standpoint, rather than one of engineering for scaling the Internet
up to the next order of magnitude or within the bounds that are likely to be
provided by Moore's Law and advancing router architectures. Also, they mainly
consider shortest path routing rather than policy routing, which is of less
interest in the context of interdomain routing on the Internet.

\section{Non-technical solutions to routing table growth \& deaggregation}

Most efforts dedicated to the problems of routing table growth and
deaggregation have focused on technical improvements to single aspects of BGP
routers or full-scale, clean slate shifts to interdomain routing architectures
with more favorable scaling properties. However, a few efforts have considered
non-technical means to manage the growth of the routing table, including
routing economies (effectively internalizing the negative externality) and
encouraging more effective network operations.

In the face of routing table growth, Rekhter et al. \cite{Rekhter:1997mi} argue
for charging neighbors for use of slots in the routing table by establishing
bilateral contracts for carrying route advertisements. They discuss the
success of what they term the ``spirit of cooperation'' in operating the
Internet thus far, but claim that it will not scale as the Internet grows
and becomes more commercial, and also claim that it will become more
difficult to come to agreement about policy governing appropriate behavior.
In place of norms, they assert that the use of financial settlement allows
for flexibility in route selection and advertisement based on the value of
selecting non-default routes. As discussed previously, there are
varying degrees of private benefit that are derived from route
deaggregation, and financial settlement creates incentives to reduce
non-valuable uses and allow for valuable uses to be compensated
appropriately. Rekhter's proposal is complex, requiring bilateral
contracts to be negotiated and executed between each pair of ASes that
exchange routes between each other. Such complexity would likely hinder
adoption, and especially incremental adoption, by ISPs. However, if
successful such a contracting system should indeed incentivize route
aggregation and efficient route announcements. Ideally, in a
competitive market, the price would begin to reflect the full cost of
advertising a prefix into the global routing table.

A more recent consideration of use of economic disincentives to discourage
routing table growth is found in \cite{Clayton:2010bh}. In this paper, Clayton
describes the challenges in appropriately charging for routing table slots, as
well as the realization that to fairly apportion his estimated marginal cost of
\$77,000 per route, some kind of mechanism to appropriately share fees with all
networks burdened by the route is required---something that would likely amount
to Rekhter et al.'s solution.

Others, such as Zhao et al. \cite{Zhao:2001ly}, briefly discuss potential
alternative business models for ISPs where cost settlements would be paid to
advertise prefixes to DFZ providers, creating an economy for routing prefixes
in addition to traffic. However, this proposal is dismissed relatively quickly
for reasons of complexity and political feasibility with the Internet
community.

Like the CIDR Report emerged from the community to inform network operators
about how to better operate their networks, other efforts have also come from
the community. First, in terms of education, presentations such as
\cite{Steenbergen:2010nx} are relatively common at NANOG meetings, and
reiterate to the community the problems associated with routing table growth
while also providing operational solutions to reduce the impact of one's
network on the routing table, such as advertising traffic engineering
routes with a NO\_EXPORT community attribute to keep TE routes out of the
routing table.  More interestingly, a volunteer effort called the CIDR
Police \cite{Nussbacher:2003ys} that emerged from the NANOG community acted
both in an educational role and as a more direct or explicit sanction for
operators to improve their behavior. The instigators of the project claim
to have had a material impact on the growth of the routing table in the
2000-2002 period.

\section{The Internet operations community and the social forces within}

There is unfortunately a dearth of work available discussing the community of
Internet creators and operators, and its norms and characteristics or the
makeup of its membership, in spite of the arguable importance of this group in
the realization and continued functioning of the Internet. Some insights can be
gathered from a thesis \cite{Mathew:2009dz} and a follow-on paper
\cite{Mathew:2010ly} that studies the social interactions between network
operators in the context of BGP operations. In these works, Mathew finds that
operators organize loosely based on reputation and network stature in the
Internet topology, and have traditionally expected other ``clueful'' operators
to operate their networks competently.

Broader discussions of the Internet ethos and the nature of the
(academically-oriented) community that was involved in the founding and growth
of the ARPANET and NSFNET, the networks that later became the Internet as we
know it today, can be found in some of the historic and ethnographic accounts
of the Internet's origins. First among the work in this area is Abbate's
\emph{Inventing the Internet} \cite{Abbate:2000ve}, and Hafner's slightly less
formal \emph{Where Wizards Stay up Late} \cite{Hafner:1996qf} was also helpful.

Finally, the nature of the Internet operations community can be gleaned to some
extent by reading the mailing lists that the members of the community
participate in, such as \cite{NANOG}.

\section{\fbox{TODO:Collective action in protocol adoption}}

\fbox{\parbox{6in}{it seems like this the discussions of incentives for
incremental adoption of protocols and ``going first'', especially with security
protocols, might be useful for this discussion, but I'm leaving this here for
now and may not return to it...}}
