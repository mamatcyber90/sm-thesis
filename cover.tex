% -*-latex-*-
% $Log: cover.tex,v $
% Revision 1.8  2008/05/13 15:02:15  jdreed
% Degree month is June, not May.  Added note about prevdegrees.
% Arthur Smith's title updated
%
% Revision 1.7  2001/02/08 18:53:16  boojum
% changed some \newpages to \cleardoublepages
%
% Revision 1.6  1999/10/21 14:49:31  boojum
% changed comment referring to documentstyle
%
% Revision 1.5  1999/10/21 14:39:04  boojum
% *** empty log message ***
%
% Revision 1.4  1997/04/18  17:54:10  othomas
% added page numbers on abstract and cover, and made 1 abstract
% page the default rather than 2.  (anne hunter tells me this
% is the new institute standard.)
%
% Revision 1.4  1997/04/18  17:54:10  othomas
% added page numbers on abstract and cover, and made 1 abstract
% page the default rather than 2.  (anne hunter tells me this
% is the new institute standard.)
%
% Revision 1.3  93/05/17  17:06:29  starflt
% Added acknowledgements section (suggested by tompalka)
%
% Revision 1.2  92/04/22  13:13:13  epeisach
% Fixes for 1991 course 6 requirements
% Phrase "and to grant others the right to do so" has been added to
% permission clause
% Second copy of abstract is not counted as separate pages so numbering works
% out
%
% Revision 1.1  92/04/22  13:08:20  epeisach

% NOTE:
% These templates make an effort to conform to the MIT Thesis specifications,
% however the specifications can change.  We recommend that you verify the
% layout of your title page with your thesis advisor and/or the MIT
% Libraries before printing your final copy.
\title{Tragedy of the routing table: An analysis of collective action amongst Internet network operators}

\author{Stephen Robert Woodrow}
% If you wish to list your previous degrees on the cover page, use the
% previous degrees command:
\prevdegrees{B.Sc. Computer Engineering, University of Manitoba (2007)}
% You can use the \\ command to list multiple previous degrees
%       \prevdegrees{B.S., University of California (1978) \\
%                    S.M., Massachusetts Institute of Technology (1981)}
\department{Engineering Systems Division \and Department of Electrical Engineering and Computer Science}

% If the thesis is for two degrees simultaneously, list them both
% separated by \and like this:
% \degree{Doctor of Philosophy \and Master of Science}
\degree{Master of Science in Technology and Policy \and Master of Science in Electrical Engineering and Computer Science}

% As of the 2007-08 academic year, valid degree months are September,
% February, or June.  The default is June.
\degreemonth{September}
\degreeyear{2011}
\thesisdate{June \colorbox{yellow}{XX}, 2011}

%% By default, the thesis will be copyrighted to MIT.  If you need to copyright
%% the thesis to yourself, just specify the `vi' documentclass option.  If for
%% some reason you want to exactly specify the copyright notice text, you can
%% use the \copyrightnoticetext command.
%\copyrightnoticetext{\copyright IBM, 1990.  Do not open till Xmas.}

% If there is more than one supervisor, use the \supervisor command
% once for each.
%\supervisor{Karen R. Sollins}{Principal Research Scientist, MIT CSAIL}
%\reader{David D. Clark}{Senior Research Scientist, MIT CSAIL}

\supervisor{Karen R. Sollins}{Principal~Research~Scientist,~Computer~Science~and~Artificial~Intelligence~Laboratory}
\reader{David D. Clark}{Senior Research Scientist,
Computer Science and Artificial Intelligence Laboratory}


% This is the department committee chairman, not the thesis committee
% chairman.  You should replace this with your Department's Committee
% Chairman.
\chairman{Leslie A. Kolodziejski}{Professor of Electrical Engineering \\ Chair, Department Committee on Graduate Students}
\director{Dava J. Newman}{Professor of Aeronautics and Astronautics and Engineering Systems \\ Director, Technology and Policy Program}

% Make the titlepage based on the above information.  If you need
% something special and can't use the standard form, you can specify
% the exact text of the titlepage yourself.  Put it in a titlepage
% environment and leave blank lines where you want vertical space.
% The spaces will be adjusted to fill the entire page.  The dotted
% lines for the signatures are made with the \signature command.
\maketitle

% The abstractpage environment sets up everything on the page except
% the text itself.  The title and other header material are put at the
% top of the page, and the supervisors are listed at the bottom.  A
% new page is begun both before and after.  Of course, an abstract may
% be more than one page itself.  If you need more control over the
% format of the page, you can use the abstract environment, which puts
% the word "Abstract" at the beginning and single spaces its text.

%% You can either \input (*not* \include) your abstract file, or you can put
%% the text of the abstract directly between the \begin{abstractpage} and
%% \end{abstractpage} commands.

% First copy: start a new page, and save the page number.
\cleardoublepage
% Uncomment the next line if you do NOT want a page number on your
% abstract and acknowledgments pages.
% \pagestyle{empty}
\setcounter{savepage}{\thepage}
\begin{abstractpage}
% $Log: abstract.tex,v $
% Revision 1.1  93/05/14  14:56:25  starflt
% Initial revision
% 
% Revision 1.1  90/05/04  10:41:01  lwvanels
% Initial revision
% 
%
%% The text of your abstract and nothing else (other than comments) goes here.
%% It will be single-spaced and the rest of the text that is supposed to go on
%% the abstract page will be generated by the abstractpage environment.  This
%% file should be \input (not \include 'd) from cover.tex.
In this thesis, I designed and implemented a compiler which performs
optimizations that reduce the number of low-level floating point operations
necessary for a specific task; this involves the optimization of chains of
floating point operations as well as the implementation of a ``fixed'' point
data type that allows some floating point operations to simulated with integer
arithmetic.  The source language of the compiler is a subset of C, and the
destination language is assembly language for a micro-floating point CPU.  An
instruction-level simulator of the CPU was written to allow testing of the
code.  A series of test pieces of codes was compiled, both with and without
optimization, to determine how effective these optimizations were.

\end{abstractpage}

% Additional copy: start a new page, and reset the page number.  This way,
% the second copy of the abstract is not counted as separate pages.
% Uncomment the next 6 lines if you need two copies of the abstract
% page.
% \setcounter{page}{\thesavepage}
% \begin{abstractpage}
% % $Log: abstract.tex,v $
% Revision 1.1  93/05/14  14:56:25  starflt
% Initial revision
% 
% Revision 1.1  90/05/04  10:41:01  lwvanels
% Initial revision
% 
%
%% The text of your abstract and nothing else (other than comments) goes here.
%% It will be single-spaced and the rest of the text that is supposed to go on
%% the abstract page will be generated by the abstractpage environment.  This
%% file should be \input (not \include 'd) from cover.tex.
In this thesis, I designed and implemented a compiler which performs
optimizations that reduce the number of low-level floating point operations
necessary for a specific task; this involves the optimization of chains of
floating point operations as well as the implementation of a ``fixed'' point
data type that allows some floating point operations to simulated with integer
arithmetic.  The source language of the compiler is a subset of C, and the
destination language is assembly language for a micro-floating point CPU.  An
instruction-level simulator of the CPU was written to allow testing of the
code.  A series of test pieces of codes was compiled, both with and without
optimization, to determine how effective these optimizations were.

% \end{abstractpage}

\cleardoublepage

\section*{Acknowledgments}
\begin{singlespace}
%This thesis marks the culimation of my journey at MIT---arguably not the smoothest (or most direct) of sails. While often challenging, my time at MIT has also been stimulating and broadening, and has caused me to grow in ways that I am not sure I would have needed to or been able to elsewhere. I am grateful for the opportunity I have had to deeply study and grow even more excited about the Internet---a space where the interplay between technology, policy, and business is incredibly rich and interesting. Participating in network operator communities, talking with experts directly, reading papers, writing code, and analyzing data have all contributed to this interest and excitement, and to this thesis. Relatedly, I am also grateful for this opportunity to get the CS education that I ``missed'' during my undergraduate computer engineering career, while also learning about other ways to think about and address the world, its insitutions, and its challenges through TPP.

While probably contributing somewhat to what I perceived to be somewhat rough seas of my graduate school career here, I should ultimately thank my adivsors Karen and Dave for giving me the flexibility and freedom to spend my time fully on what would eventually and somewhat unexpectedly become my thesis research. You might say I was given enough rope to hang myself, but thankfully, with their help, I managed to avoid such a fate. With this long leash and the problems that inevitably occur when doing things for the first time, I gained an appreciation for what it really means to do research for a living, including the perhaps most important first steps of picking a good question to answer and being excited about the work. I also learned how important it is to talk to people about your work, for feedback, criticism, new ideas, or in the case of Internet operators, an understanding of how the Internet actually works, rather than how we might learn about it at the university.

%More concretely, in terms of this project, I am glad that I got to use data, but lament how the presence of so much data (hundreds of gigabytes) diverted my attention to building tools to process, manipulate, and anlyze the data instead of spending enough time up front writing and presenting the information in this thesis. Done is better than perfect, and it's hard to be perfect

\vspace{1em}

Finally (last but not least), the obligatory round of thanks for and acknowledgement of those who influenced me during this effort:

My first and biggest thanks are to Karen Sollins, my adivsor. She first helped me by hiring me when I was searching for an RA position related to my interests in the Internet in November of 2008, and has been a constant source of encouragement, understanding, and good advice on my career as a graduate student since then. Karen also took efforts to involve me in a number of interesting projects and has offered much feedback and helpful perspectives on this work, for which I am greatly appreciative.

I would also like to thank David Clark for his feedback and advice as he became involved in my project as I changed to focus on what would ultimately become this thesis. His experience with and insights about Internet policy and economics together with his technical knoweldge was very helpful, as was his suggestion to use Elinor Ostrom's CPR framework to think about the Internet routing table.

Steven Bauer and I first met when I was searching for an RA, but we ultimately became better friends when he returned to MIT to work on another project. I would argue that Steve has helped keep me sane, as he has been a great sounding board, a source of strong feedback, and an advocate, as well as someone to ``geek out'' with on topics such as Python, TCP, and startups. Thanks for all these things, Steve.

Arthur Berger has been invaluable in helping me plan my investigation and analysis and offering suggestions and assistance in interpreting the resulting data, for which I am immensely grateful. Arthur and I also had several discussions about my professional aspirations that were helpful in my deciding what to do next, and I appreciate his thoughts and advice to this end.

I would like to thank a number of members of the Internet community for volunteering to talk with me about some of the various questions I've had about the Internet. In particular, these individuals' insights and opinions were extremely valuable in understanding how the Internet works \emph{in practice}, which is often quite different from theory and academic viewpoints. Thank you Martin Hannigan, Geoff Huston, Tony Li, Bruce Davie, and Jeff Schiller.

Thanks to Susan Perez, the ANA administrative assistant, for always being incredibly helpful and cheerful. I must acknowledge my officemates in 32-G806 for always providing good company and discussion---especially Jesse Sowell and Chintan Vaishnav---who have offered research advice and discussion when asked, and passionate conversation always.

Thanks are due to my parents and sister as well, and in particular my mother, for their support and encouragement through this process. It started with the drive down to Massachusetts in 2008 and has continued to the present with the Calvin and Hobbes comic that has graced my inbox daily since April.

Finally, I acknowledge the financial support of Intel, the MIT Communications Futures Program, and the Natural Sciences and Engineering Research Council of Canada (NSERC).
\end{singlespace}
%%%%%%%%%%%%%%%%%%%%%%%%%%%%%%%%%%%%%%%%%%%%%%%%%%%%%%%%%%%%%%%%%%%%%%
% -*-latex-*-
