\chapter{Analyzing the CIDR Report}

This chapter describes the methods I employed to analyze the effectiveness of the CIDR Report. I begin with an overview of the analytical approach I designed and undertook, followed by the specific details of how I collected data and built the analytical tools to generate the data set that I analyze in the following chapter.

\section{Analytical Approach}

% be sure to avoid value-laden terminology within this

The general intuition in analyzing the effectiveness of the CIDR Report is relatively simple: \emph{do autonomous systems change their behavior, as measured by the number of aggregable routes they advertise into the routing table, after appearing on the CIDR Report?} If the hypothesis that the CIDR Report was effective in controlling routing table growth based on the social forces or reputation in the Internet operator community holds, then the behavior of ASes that appear on the CIDR report should differ---becoming more aggregated---compared to the behavior of ASes that never appear on the CIDR Report. This can be viewed as a quasi-experiment \cite{babbie?}, with ASes that appear on the CIDR report composing the treatment group and ASes that never appear as the control group. This cannot be viewed as a true natural experiment or controlled experiment because the ASes in the treatment group are not randomly selected. The behavior that leads to appearance on the CIDR report is typical of large ISPs and so they are disproportionally represented on the CIDR Report. This non-random appearance of ASes in the treatment group raise potential validity concerns, and the implications of this are discussed later.

The analysis of the performance of the treatment group could be implemented very simply by comparing the behavior of a given AS over time, starting when it first appears on the top thirty aggregation report. The CIDR report emails transmitted to mailing lists weekly contain much of the information necessary to conduct this analysis. Accordingly, I first gathered and coded the data contained in these emails from network operator mailing list archives. This information determines which ASes are in the treatment group and when they appeared.

The CIDR Report emails have two shortcomings demand the gathering of additional data about the routing table. The aggregation report contains no information about ASes that do not appear within the top thirty, requiring other information to be gathered in order to form a control group. Also, rank (and thus appearance) is determined by relative ranking rather than absolute number of routes, and so an AS in the treatment group may leave the top 30 without aggregating simply because other ASes deaggregate more. In order to measure behavior in terms of routing table slots consumed by an AS, a non-truncated version of the CIDR report is required. No archives of the full report are available, so I instead gathered historic routing table data and developed my own implementation of the aggregation report. The information from this generated aggregation report provides the metrics used for analysis of ASes in both the control and treatment groups for consistency.

Finally, utilizing the information from both the emailed (authoritative) CIDR report and the full aggregation report generated from historic routing table data, I proceed with analysis, characterizing the overall behaviorof ASes in the treatment and control groups, as well as behavior of ASes in both groups over time in order to observe whether the CIDR report may have had an effect on treated AS' behavior.

\section{Data sources \& data collection} %%%%%%%%%%%%%%%%%%%%%%%%%%%%%%%

\subsection{Authoritiative CIDR Reports}

The official CIDR Report has been transmitted weekly on Friday afternoons to network operator communities including the North American Network Operators Group (NANOG), starting in September 1997. The NANOG community has kept a public archive, \cite{NANOG-Archives} and \cite{NANOG-NEW-Archives}, of all messages sent to its mailing list since its inception in 1994, capturing all of the CIDR report messages in its archive. This archive is an appropriate source of CIDR Reports as it contains the messages that were actually received by network operators (and so used to apply social force under the CIDR Report hypothesis).

The indexes of the archives were downloaded and parsed to obtain a set of message headers containing the sender name, subject, and date of all messages sent to the NANOG list. This message list was then filtered to create a coding candidate set containing only messages with a subject line containing "CIDR R" in any case.

The full bodies of the messages in the candidate set were downloaded from the NANOG archives for coding. The coding process consisted of viewing the message and classifying it as one of:

\begin{itemize}
\item{an official CIDR Report that appears to be correct,}
\item{an official CIDR Report that appears incorrect (duplicates, sent on the wrong date, containing obviously invalid data, etc.),}
\item{an email from the operator community praising, criticizing, or discussing behavior in the CIDR Report, or}
\item{none of the above (not of interest).}
\end{itemize}

Following the coding process, emails coded as official and correct CIDR reports were parsed by an automated program to extract rank prefix count information for every AS appearing on the aggregation report. This information was then stored in a database for later use by other analysis tools.

One benefit of coding the authoritative CIDR report emails by hand was that it enabled observation of changes in the CIDR report's BGP data vantage point and implementation over time, as well as the transition of stewardship of the report. The report as originally conceived and implemented by Tony Bates was used from the first NANOG CIDR Report on 17 September 1996 until 23 August 2002. Geoff Huston then took responsibility for the report on 30 August 2002, producing a similarly-formatted report but with a new implemenation that was developed without consulting Bates' source code \cite{Huston interview}. The CIDR report's vantage points over time are in the table below. The potential importance of vantage point selection is discussed further in chapter \fbox{Discussion}.

\begin{table}[h]
    \begin{center}
    \caption{CIDR Report vantage point ASes over time}
    \vspace{1em}
    \begin{tabular}{c | c | c}
        AS number & AS Name & Date effective \\
        \hline
        AS 5413 & Xara.net (at MAE-East) & 17 September 1996 \\
        AS 5413 & GX Networks & 15 June 2001 \\
        AS 6447 & Route Views & 30 August 2002 \\
        AS 4637 & REACH & 11 October 2002 \\
        AS 2.0  & APNIC R\&D & 20 July 2007 \\
    \end{tabular}
    \end{center}
\end{table}

\subsection{Routing Table Data}

The University of Oregon Route Views project \cite{RouteViews} was used as the source of routing table data for generating a full aggregation report. Route Views gathers multiple views of the Internet routing table as seen by various major providers that peer with Route Views, and is a commonly used data source for operational and academic research. Route Views peers change over time, but typically include most of the default-free zone (DFZ) providers. The project has been storing routing table archives since November 1997.

Route Views' data sets export the entire Route Views RIB which contains all routes received from each provider it peers with. This differs from the Huston CIDR Report implementation \cite{Huston interview} (and possibly from the original Bates CIDR Report implementation also), which used the best BGP route available instead of all routes available.

%The multitude of vantage points included in the Route Views data is arguably preferable in that it allows more potential aggregation to be visible to the CIDR Report algorithm, as discussed in the background section. It is also potentially problematic in that it includes ``leaked'' routes---routes not intended to be advertised globally---if they are advertised by even one provider. Discussion of the variation in routes observed from peers and the consequences of this---the potential importance of vantage point---are discussed later.

Route Views RIB snapshots are typically generated every two hours, and so a script was developed to search the archive indexes for the most appropriate snapshot to download in accordance with the CIDR Report's weekly Friday release schedule. The script selects the RIB snapshot created most recently after midnight each Friday. For the cases where RIB files were not found for a given Friday, the most recently generated RIB from earlier in the same week was selected instead. Between November 1997 and November 2001, snapshots in ``Cisco CLI''-format RIBs were downloaded from \url{route-views.routeviews.org}. After November 2001, MRT-format RIBs were downloaded from \url{route-views2.routeviews.org}. These RIBs were stored for preprocessing and ultimately the generation of the full aggregation report.

\section{Data preprocessing}

Several preprocessing steps are required to extract and normalize the data from the Route Views RIB files in order to prepare the data for use in generating the aggregation report.

First the information required from each RIB snapshot---namely the advertised prefix, the IP address of the observing peer, and the AS-PATH associated with the route---are extracted from the snapshot files using RIPE's libbgpdump library \cite{libbgpdump} for MRT-format RIB snapshots and CAIDA's straightenRV tool \cite{straightenrv} for ``Cisco CLI''-format RIB snapshots. The output of these programs was a plain text representation of the information extracted from the RIB.

Next, these simplified RIB files were processed to canonicalize the AS-PATH associated with each prefix. Since AS-PATH equality is used by the aggregation report to determine whether the routing policy of two prefixes is the same, a canonical representation of the AS-PATH is essential in order to allow AS-PATH comparison. There are four steps in the canonicalization process:

\begin{enumerate}
\item{Removal of non-AS\_SEQUENCE elements in the AS\_PATH. AS\_SETs (denoted as \verb!{4,5}! for a set containing AS 4 and 5) can make the origin of a route ambiguous because there is not a single AS in the origin position of the AS\_PATH. To resolve such ambiguity, any AS\_SET is reduced to a single AS number if it contains only one AS number, or is discarded if it contain multiple unique AS numbers. AS\_CONFED\_SEQUENCEs (denoted as \verb![4 5]!) and AS\_CONFED\_SET (denoted as \verb!(4,5)!) should only contain private AS numbers and so are discarded without further consideration.

As an example of this transformation, the path \verb!1 1 1 2 3 (65535,65533) {4,4}! would become \verb!1 1 1 2 3 4!.
}

\item{Removal of private AS numbers. IANA has allocated private AS numbers \cite{IANA AS number list} for ISP internal use and documentation purposes, and these AS numbers sometimes appear in the Internet routing table. AS numbers within this range are removed from the AS\_PATH.}

\item{Removal of prepended AS numbers. As discussed earlier, AS\_PATH prepending is sometimes used by network operators to make a route appear less attractive to the BGP path selection algorithm. A prepended AS\_PATH affects traffic engineering but not routing policy, and so prepended AS numbers are removed by collapsing contiguous blocks of the same AS number into a single entry in the AS\_PATH.

For example, the path \verb!1 1 1 2 3 3 4! would be collapsed to \verb!1 2 3 4!.}
\item{Removal of simple routing loops. Minor routing loops sometimes occur in the BGP routing table, such as when a route traverses a provider that uses multiple AS numbers for different parts of their infrastructure. These loops are removed following the algorithm used by CAIDA in straightenRV\cite{straightenrv}. If a more complex loop is found that cannot be resolved using this algorithm, the entire route is discarded.

For example, the path \verb!1 2 3 2 5!, which may have resulted from AS 2 and 3 belonging to the same operator, is reduced to \verb!1 2 5!. In contrast, a more complex loop such as \verb!1 2 3 2 3 4! cannot be resolved using CAIDA's heuristic.}
\end{enumerate}

Finally, because this canonicalization process requires a full traversal of the input RIB in order to produce the canonical RIB, other data sets can be opportunistically generated during the canonialization process. The number of prefixes observed by each Route Views peer are recorded, for each origin AS and for the entire routing table, to allow later analysis of variation in the prefixes seen per peer. This information was a helpful debugging tool and will also be used for the later discussion of the consequences of vantage point selection.

\section{Implementing the Aggregation Report}

Determining which prefixes in a RIB are aggregable consists of three major steps. First, a prefix tree data structure containing all advertised prefixes is constructed in order to establish which prefixes are adjacent to and covered by other prefixes. Next, the tree is walked recursively aggregate adjacent prefixes and to classify prefixes that may be aggregated by a covering prefix, both only in cases where routing policy is not compromised. Finally, counts of aggregable and total prefixes are attributed to each origin AS. These counts are then ranked to generate the aggregation report as seen on the CIDR Report. Each of these steps is described in more detail below.

To construct the binary prefix tree containing a set of prefixes that we wish to aggregate over, we must first determine the root of the tree. If prefixes of any length were allowed then the tree would need to be rooted at 0.0.0.0/0. However, because IANA has always allocated IP address blocks as Class A or /8 blocks, we can generate the aggregation report by considering each /8 separately. Taking advantage of this fact will make the implementation of the classification algorithm more efficient as the entire routing table need not be stored in memory. Accordingly, the input RIB will be sorted on the first octet of the prefix and processed by the algorithm one /8 at a time.

For each /8, a prefix tree will be constructed with the root at the the /8 prefix (for example, 10.0.0.0/8). Then each prefix in the RIB, as observed by each peer, is inserted into the tree. Each node contains a prefix, the corresponding AS\_PATH for that prefix, and pointers to the two more specific sub-prefixes of the current prefix if they exist. An small example of a prefix tree is shown in Figure \ref{fig:ex_prefix_tree}.

\begin{figure}
    \caption{A prefix tree for 10.0.0.0/8 and some longer prefixes}
    \label{fig:ex_prefix_tree}
\end{figure}

To insert a prefix into the tree, a cursor is placed at the root and then advanced to one of the two children of the root depending on whether the first bit after the first octet of the IP prefix is a `1' or a `0'. This process is then repeated recursively from the current node for each subsequent bit in the prefix. If at any point a node does not have a child must be traversed to reach the insertion point, a placeholder prefix is inserted to maintain the tree structure. When the depth in the tree is equal to the length of the prefix less the original eight bits of the /8, the prefix is inserted into the tree. The insertion process is illustrated in Figure \ref{fig:ex_prefix_tree_insert}.

\begin{figure}
    \caption{Insertion of 10.112.0.0/12 into the previously described prefix tree.}
    \label{fig:ex_prefix_tree_insert}
\end{figure}

Unlike the Huston and likely the Bates implementations of the aggregation report, this implementation considers all routes available from all peers for a given prefix in the routing table, instead of just the best route. Thus, for a given prefix in the tree, there are actually multiple AS\_PATHS stored---one for each peer that observes the route. This can be viewed logically as several prefix trees overlaid on top of eachother, though the implementation uses a single tree with multiple AS\_PATHs per prefix. All distinct AS\_PATHs for a given prefix are gathered from the input RIB and associated with the prefix before it is installed in the prefix tree.

%This is important in that it allows the most potential aggregation to be observed because it does not conflate routes viewed from different peers. For example, consider the following topology shown in Figure \ref{fig:ex_topology}, and the routes observed from two peers shown in the routing table below:
%
%\begin{figure}
%\caption{Example topology and route announcements.}
%\label{fig:ex_topology}
%\end{figure}
%
%% \begin{tabular}{l | l}
%%     prefix        & AS\_PATH \\
%%     \hline
%%     10.0.0.0/8    & 30 20 10 \\
%%     10.0.0.0/9    & 30 20 10 \\
%%     10.128.0.0/9  & 30 20 10 \\
%% \end{tabular}
%%
%% \begin{tabular}{l | l}
%%     prefix        & AS\_PATH \\
%%     \hline
%%     10.0.0.0/8    & 40 10       \\
%%     10.0.0.0/9    & 40 30 20 10 \\
%%     10.128.0.0/9  & 40 30 20 10 \\
%% \end{tabular}
%%
%% \begin{tabular}{l | l}
%%     prefix        & AS\_PATH \\
%%     \hline
%%     10.0.0.0/8    & 40 10    \\
%%     10.0.0.0/9    & 30 20 10 \\
%%     10.128.0.0/9  & 30 20 10 \\
%% \end{tabular}
%
%\begin{figure}
%\caption{Example routing tables based on routes and topology in figure \ref{fig:ex_topology}}
%\label{fig:ex_tables}
%\begin{centering}
%\begin{tabular}{l | l c l | l c l | l}
%    \multicolumn{2}{c}{\textsc{Routes From AS 30:}} & & \multicolumn{2}{c}{\textsc{Routes From AS 40:}} & & \multicolumn{2}{c}{\textsc{Observer's Best Routes:}} \\
%    prefix        & AS\_PATH & & prefix       & AS\_PATH    & & prefix & AS\_PATH \\
%    \cline{1-2} \cline{4-5} \cline{7-8}
%    10.0.0.0/8    & 30 20 10 & & 10.0.0.0/8   & 40 10       & & 10.0.0.0/8 & 40 10 \\
%    10.0.0.0/9    & 30 20 10 & & 10.0.0.0/9   & 40 30 20 10 & & 10.0.0.0/9 & 30 20 10 \\
%    10.128.0.0/9  & 30 20 10 & & 10.128.0.0/9 & 40 30 20 10 & & 10.128.0.0/9 & 30 20 10 \\
%\end{tabular}
%\end{centering}
%\end{figure}
%
%By looking at only the BGP-selected best routes from the observation vantage point (the right-most table in Figure \ref{fig:ex_tables}), it's possible to conclude that routes aren't aggregable.



%- sort incoming data file on first octet of prefix and also to cluster prefixes into contiguous blocks
%	- take advantage of RIR/IANA allocation policies of nothing larger than a /8 to make implementation more efficient
%- for a given /8, construct a binary search tree representing the state of each bit in the prefix
%	- explain binary tree, and given examples
%	- process all routes for a given prefix, adding them as attributes to that prefix
%	- insert prefix into a binary tree
%	- explain how there are effectively many overlaid binary trees, one for each peer that sees a given prefix, which is relevant for the classification part

Once all of the prefixes in the RIB for the current /8 are inserted into the prefix tree, the prefix tree is walked recursively to aggregate and classify aggregable prefixes within the tree. As noted in the background, a prefix is considered aggregable if it has the same routing policy as a less-specific covering prefix. The aggregation and classification algorithm was implemented based on the description of the CIDR Report included in the Report's preamble, as well as detailed records of the report's operation \cite{CIDR Report details}. Like the authoritiative CIDR Report, routing policy equality in this implementation of the aggregation report is determined by comparing AS\_PATH equality.

The general algorithm behind the aggregation and classification process is as follows. The prefix tree is traversed using post-order recursion and the following operations are performed on each node before the function returns to its calling parent (the post-order traversal ensures that children are aggregated before their parents):
\begin{enumerate}
\item{Attempt to aggregate children: If the current prefix is a placeholder and has two more-specific (child) prefixes announced in the routing table, check to see if their AS\_PATHs match. If they do, convert the current prefix into a ``true'' (non-placeholder) prefix and mark the children prefixes as aggregable.}
\item{Attempt to aggregate by a covering prefix: Compare the AS\_PATH of the current prefix with the AS\_PATH of its nearest less-specific (ancestor) prefix . If they match, mark the current prefix as aggregable.}
\end{enumerate}

%A small example is show in figure
%
%\begin{figure}
%\label{fig:ex_aggregation}
%\caption{An example of the aggregation process}
%\end{figure}

This process is again logically conducted as though there are multiple separate but overlaid trees for each Route Views peer/observer AS. The process is actually implemented by processing all vantage points available at a given prefix and aggregating and classifying them against other ancestor and child prefixes visibile from the same peer AS. Classifications of aggregability are made on a per-peer basis, and then must be generalized for the entire prefix. There is no single way to make this generalization and so a design decision must be made.

There are two obvious choices for how generalize across each vantage point's aggregation classification for a given prefix: consider the prefix aggregable if \emph{any} vantage point considers it aggrebable, or only if \emph{all} vantage points consider it aggregable. The latter option requires consensus that a prefix be aggregable across all views, while the former allows any claim of aggregability to stand. This implementation of the aggregation report classifies a prefix as aggregable if it is classified as aggregable from \emph{any} vantage point. While perhaps ``pessimistic'', this approach will detect and report the maximum amount of aggregation possible.

Finally, when a prefix is marked as aggregable, it must be attributed to the network that announced the prefix as this network is responsible for the redundant route announcement. This is normally performed by attributing the aggregable route to the origin AS---the first AS in the route's AS\_PATH---that is presumed to be the announcing network. An ambiguous situation arises in the case of prefixes announced by multiple ASes, a condition known as a multiple origin AS (MOAS) \cite{moas-10.1145/505202.505207}. Again, a design decision must be made for how to attribute aggregable MOAS prefixes: the MOAS prefixes could be ignored, the route's behavior could be attributed to one of the origin ASes, or it could be attributed to all origin ASes. This implementation attributes an aggregable prefix to each unique AS that announces the prefix.

To attribute deaggregation behavior to each AS, the aggregation report processor maintains three quanities for each origin AS:

\begin{itemize}
\item{Announced routes: routes that were originally announced in the input RIB. This is the ``netsnow'' quantity in the aggregation report}
\item{Withdrawn routes: routes that were covered by less specific routes and thus would be withdrawn from the ideally-aggregated routing table.}
\item{Aggregated routes: routes that did not exist in the original routing table, but were synthesized by combining two adjacent prefixes into a covering prefix.}
\end{itemize}

If a prefix in the prefix tree is marked as aggregable and is originally from the input RIB, then this prefix is counted as a withdrawn route. If a prefix is synthetic aggregate (not from the input RIB) and not marked as aggregable, then it is counted as an aggregated route. These quantities are then used to calculate the ``netgain'' quantity for a given AS.

After all of the RIB's routes are processed, the counts of aggregable routes attributed to their origin ASes---in the form of a list of tuples (origin AS, netsnow, netgain, netsaggr, aggregated, withdrawn)---must be sorted to determine the ranking of networks as found on the authoritative CIDR Report. The ranking of ASes on the aggregation report is generated by a primary sort on the netgain value, the number of aggregable prefixes announced by the AS. The AS with the greatest netgain value will be hold rank 1 on the aggregation report. To provide an ordering when networks have the same netgain value, a secondary sort is performed to rank networks with a lesser netsnow value above networks with a greater netsnow value if both have the same netgain---these networks are more deaggregated as a fraction of their total routes. Finally, a tertiary sort is performed on the AS number as a sort of last resort to produce a canonical ordering similar to that found on the full, authoritative CIDR Report \cite{CIDR Report full ranking}. With this final sorting, the aggregation report data is ready for analysis.

\section{Analyzing the Aggregation Report}



% Handling gaps in the data
%    continue them until the end of the gap
% Extract data from CIDR report according to intitial selection criteria
% Define and determine appearances from CIDR Report data
%    **NOT** four weeks on the report -- whooops
%    maximum gap of 8 weeks -- avoid multiple appearances
%    also remove overlap by amalgamating appearances that start within two years of eachotehr
% Select control group candidates and establish ``appearances'' within the control group
%    minimum
% For both the treatment and control group, measure behavior at various time intervals out from the initial appearaance


% valdiity notes
% - look at CDF of population elligible to be control
% cidr report overall behavior notes
% -> SELECT * FROM get_cumulative_ecr_asns();
% -> SELECT * FROM cumulative_gcr_as_counts;


%////////////////////////////////////////////////////////////////////////


%
%ANALYTICAL METHODS
%
%////////////////////////////////////////////////////////////////////////////////
%////////////////////////////////////////////////////////////////////////////////
%////////////////////////////////////////////////////////////////////////////////
%
%ANALYSIS
%
%- Available data/data overview
%	- plot 'h' plots of available data
%	- plot total prefixes in the total table (Routeviews) and in the report (Cidr Report) to see that there aren't any discontinuities
%
%- Accuracy/error of my implementation of the CIDR report
%
%- Visualization of aggregate behavior and general characteristics (in particular stratification at the top of the CIDR report)
%    - Definition of metrics used in the analysis
%        - rank on CIDR Report (order by netgain)
%        - absolute/delta netgain
%        - relative/delta netgain (relative to netsnow)
%    - CDFs by AS appearnace
%    - CDFs by rank
%
%    - Summary/general statistics and characteristics of behavior over time (time series?), across and within groups, etc. (p62)
%
%- AS Behavior after appearing on the CIDR Report
%	- Distribution of AS behaviors from T=0, T=+1 month, 3 m, 6m, 1y, 2y, etc.
%		- all appearances on the cidr report
%		- no appearance/randomly selected control
%		- appearances that actually had behavior changes observed
%	- slice by date (NOT by rank on CIDR report), and by netsnow
%
%- Variance in views of data from different peers
%- for different origin ASes that appear on the CIDR Report
%- across the entire routing table
%
%////////////////////////////////////////////////////////////////////////////////
%////////////////////////////////////////////////////////////////////////////////
%////////////////////////////////////////////////////////////////////////////////
%
%DISCUSSION
%
%Discussion of the design assumptions, and the sensitivity of the CIDR report to those assumptions
