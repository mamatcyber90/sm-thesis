This thesis marks the culimation of my journey at MIT---arguably not the smoothest (or most direct) of sails. While often challenging, my time at MIT has also been stimulating and broadening, and has caused me to grow in ways that I am not sure I would have needed to or been able to elsewhere. I am grateful for the opportunity I have had to deeply study and grow even more excited about the Internet---a space where the interplay between technology, policy, and business is incredibly rich and interesting. Participating in network operator communities, talking with experts directly, reading papers, writing code, and analyzing data have all contributed to this interest and excitement, and to this thesis. Relatedly, I am also grateful for this opportunity to get the CS education that I ``missed'' during my undergraduate computer engineering career, while also learning about other ways to think about and address the world, its insitutions, and its challenges through TPP.

While probably contributing to what I perceived to be somewhat rough seas of my graduate school career here, I should ultimately thank my adivsors Karen and Dave for giving me the flexibility and freedom to spend my time fully on what would eventually and somewhat unexpectedly become my thesis research. You might say I was given enough rope to hang myself, but thankfully, with their help, I managed to avoid such a fate. With this long leash and the problems that inevitably occur when doing things for the first time, I gained an appreciation for what it really means to do research for a living, including the perhaps most important first steps of picking a good question to answer and being excited about the work. I also learned how important it is to talk to people about your work, for feedback, criticism, new ideas, or in the case of Internet operators, an understanding of how the Internet actually works, rather than how we might learn about it at the university.

% With regard to this thesis, I am ultimately happy that I settled on a data-driven project---it never ceases to be helpful to understand and talk about the real world with real data. However, I am shocked at how the clear writing of the thesis fails to convey the process undertaken (the hours of debugging, hundreds of cpu-hours of processing, or the hundreds of gigabytes of data) to ultimately arrive at this point.
% 
% Done is better than perfect, and it's hard to be perfect


%More concretely, in terms of this project, I am glad that I got to use data, but lament how the presence of so much data (hundreds of gigabytes) diverted my attention to building tools to process, manipulate, and anlyze the data instead of spending enough time up front writing and presenting the information in this thesis. Done is better than perfect, and it's hard to be perfect

\vspace{1em}

Finally (last but not least), the obligatory round of thanks for and acknowledgement of those who influenced me during this effort:

My first and biggest thanks are to Karen Sollins, my adivsor. She first helped me by hiring me when I was searching for an RA position related to my interests in the Internet in November of 2008, and has been a constant source of encouragement, understanding, and good advice on my career as a graduate student since then. Karen also took efforts to involve me in a number of interesting projects and has offered much feedback and helpful perspectives on this work, for which I am greatly appreciative.

I would also like to thank David Clark for his feedback and advice as he became involved in my project as I changed to focus on what would ultimately become this thesis. His experience with and insights about Internet policy and economics together with his technical knoweldge was very helpful, as was his suggestion to use Elinor Ostrom's CPR framework to think about the Internet routing table.

Steven Bauer and I first met when I was searching for an RA, but we ultimately became better friends when he returned to MIT to work on another project. I would argue that Steve has helped keep me sane, as he has been a great sounding board, a source of strong feedback, and an advocate, as well as someone to ``geek out'' with on topics such as Python, TCP, and startups. Thanks for all these things, Steve.

Arthur Berger has been invaluable in helping me plan my investigation and analysis and offering suggestions and assistance in interpreting the resulting data, for which I am immensely grateful. Arthur and I also had several discussions about my professional aspirations that were helpful in my deciding what to do next, and I appreciate his thoughts and advice to this end.

I would like to thank a number of members of the Internet community for volunteering to talk with me about some of the various questions I've had about the Internet. In particular, these individuals' insights and opinions were extremely valuable in understanding how the Internet works \emph{in practice}, which is often quite different from theory and academic viewpoints. Thank you Martin Hannigan, Geoff Huston, Tony Li, Bruce Davie, and Jeff Schiller.

Thanks to Susan Perez, the ANA administrative assistant, for always being incredibly helpful and cheerful. I must acknowledge my officemates in 32-G806 for always providing good company and discussion---especially Jesse Sowell and Chintan Vaishnav---who have offered research advice and discussion when asked, and passionate conversation always.

Thanks are due to my parents and sister as well, and in particular my mother, for their support and encouragement through this process. It started with the drive down to Massachusetts in 2008 and has continued to the present with the Calvin and Hobbes comic that has graced my inbox daily since April.

Finally, I acknowledge the financial support of Intel, the MIT Communications Futures Program, and the Natural Sciences and Engineering Research Council of Canada (NSERC).