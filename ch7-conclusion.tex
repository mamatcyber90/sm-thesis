\chapter{\fbox{TODO:Conclusions and Future Work}}
\label{chap:conclusion}

\section{Is routing table growth still a problem?}
blah

%Is routing table growth still a problem? If you conclude yes, then by all accounts it will be more difficult to coordinate our way out of the next routing table growth ceiling. When you compare BGP3->BGP4 (although they were somewhat compatible) to IPv4->IPv6, the progress on the former is relatively incredible. While this may be because of the relatively greater pain that was being experienced at the time (that we haven't yet seen in IPv4), it seems like a reasonable conclusion all the same.


\cite{Fall:2009fk}

\section{Technical solutions to routing table growth}

ILNP may have killer apps, but not for ipv6 or routing table growth

%
%We'll need to coordinate again in some way, but how do we do that?
%- switch protocols?
%- economic solutions
%- improve social/CPR governance?
%  - need to be able to answer questions about what is "fair"? ("know it when you see it?")
%  - would something like a routing table governance council actually allow continued enjoyment of the dynamic aspects of internet routing?
%  - need sanctions (see what else ostrom has to say about this)
%    - how do you actually do this? routing, but there is a problem of adoption and market power and who has influence

\section{Non-technical solutions}

Discussion of the challenges of the changing landscape of the Internet http://portal.acm.org/citation.cfm?id=1851194

Sanctions are a challenge, and while there are some opportunities available within the routing system, the economic or operational ``damage'' that can be done is proportional to the content or customers you serve. Looking at other mechanisms for unilateral or multilateral sanctions within the Internet is a major opportunity for future work.

Economic arguments and palatability in considering replacement protocols in the IETF

Could a "cartel" of DFZ/Tier-1 providers start this because they have the disciplinary ability? We can look to the minimum prefix length requirements in the IPv4 days (more research required) and in IPv6 with the /32 minimum with Verizon, and how this got washed away by an interest in serving the market -- apparently the incentive to curtail routing table growth just isn't there...?

While the norms and values embedded in the Internet community may have been helpful in past, it's not clear that these are or will continue to hold (cite Tussle in Cyberspace) and so we probably shouldn't expect the CIDR report to work in the same way, especially with expansion into other regions where the culture/Internet ethos isn't the same, or where people aren't aware of the history, or loosely/not-at-all connected to the community that does still exist.

As ostrom points out, there's a meta-problem of supplying instutions and collective action in bringing about the instiutions one wants to exist to solve the first-order collective action problem in the first place.

The CIDR Report focuses on outliers with major aggregation problems -- 0.1\% of the routing table with 20\% of the prefixes. Is this the best approach? It works, but 80\% of the remaining prefixes come from 20\% of the population.

The CIDR Report is necessarily limited to a fixed number because of human
attention span and comprehension, as well as scare time resources to go and do
something about this---it makes superficially intuitive sense to focus on the
outliers. Yet there's a large group below the outliers that are also
deaggregating. Not tons individually, but collectively it's not insignificant.

Citadinni et al. also point out that the bad guys have stayed the same
generally, since 2001 -- again, not helpful

This suggests that automated sanctions or pricing might be more effective.

\section{Future Work}
Measure pre-treatment behavior (behavior before appearance i.e. T-30days...) to see if behavior changed, rather than
Use stronger statistical measures

Look at aggregation behavior moving down the list to see where a break occurs
