\chapter{Conclusions and Future Work}
\label{chap:conclusion}

This thesis set out to investigate whether the CIDR Report was effective in
motivating collective action to mitigate the effects of routing table growth,
and to draw conclusions about how to manage the Internet routing table. By
analyzing the route announcement and aggregation behavior of ASes after
appearing on the CIDR Report and comparing this to the behavior of ASes that
never appeared on the CIDR Report, we found that an improvement in AS
aggregation behavior was associated with appearing on the CIDR Report, and that
this treatment effect decreased in more recent time periods. The overall
routing table continued to grow in spite of this treatment effect, though this
was likely due to the organic growth and participation of new networks, rather
than continued deaggregation of CIDR Report participants. Regardless, the
routing table has not recently grown at a rate that exceeds upper bounds of our
technology (Figure \ref{fig:li_router_scalability}) or the capabilities of
routers available on the market. We now turn our attention to second objective
of the thesis, about how to manage the Internet routing table.

% \section{Returning to the problem of routing table growth and deaggregation}
% With this summary of our observations and lessons learned from studying the
% CIDR Report, we return to the broader issue of this thesis regarding
% collective action and the sustainable growth of the routing table.

\section{Is routing table growth a problem?}
The first question that must be asked in considering the issue of managing the
routing table is whether routing table growth is even still a problem affecting
the scalability of our current interdomain routing architecture. The Internet
routing table is no longer the driving force behind router capabilities
\cite{Davie:2011uq}, and as illustrated in Chapter \ref{chap:intro} and also
discussed in \cite{Fall:2009fk} and cited by \cite{Huston:2011ys}, Moore's law
has outpaced routing table growth.

However, there are also evolutionary changes taking place in the Interdomain
routing space that that may cause the Internet routing table to grow faster
than has come to be expected. The Internet will likely be using IPv4 addresses
for some time, given the slow pace at which the transition to IPv6 has taken
thus far. While the forthcoming exhaustion of free address space will likely
hasten IPv6 adoption, it will also likely incentivize the more efficient use of
IPv4 addresses in order to extend the useful lifetime of IPv4 before IPv6
adoption is deemed complete. This more efficient use will likely be
accomplished through partitioning larger address blocks for trading, as well as
the deaggregation of larger blocks into smaller announced prefixes to faciliate
expression of routing policy for larger blocks of machines behind NATs and
other transition technologies.

At the same time, as the IPv6 routing system moves from experimental and
prototype implementations to production-ready networks that more closely mirror
the IPv4 routing system \cite{Cowie:2010vn}, the number of prefixes in the IPv6
routing table will also grow to be larger than present, and indeed appears to
be growing exponentially right now (though the table itself is still quite
small\footnote{The IPv6 routing table contains approximately 6000 prefixes, but
currently appears to be growing exponentially:
\url{http://bgp.potaroo.net/v6/as6447/}}). If IPv6 is deployed via
dual-stacked routers, as is often the case, this means that available router
resources for storing routing tables and processing BGP updates will be
contended for by two routing tables instead of just one. Eventually the IPv4
table should be able to be deprecated, but it is unknown when that will be
viewed to be commercially viable---a point at which all customers have at least
as good IPv6 connectivity as IPv4 connectivity.

So while routers currently have sufficient capacity to meet ISP internal needs
while also adequately supporting the Internet routing table, the scaling
constraints of BGP that were introduced at the beginning of this thesis have
not been eliminated; Moore's law has simply allowed these limitations to be
outpaced by technology. If the growth properties of the Internet routing table
do change in future, our interdomain routing system may again be strained or
made less robust by its own growth, as there is nothing that ultimately
constrains the size of the routing table.

If we agree that there are at least potential problems of scalability or
robustness affecting our interdomain routing system under certain routing table
growth scenarios, the next question is: what should be done about this problem?
Assuming we believe the Internet is too valuable to leave to become unreliable,
then some action must be taken. A solutions space is outlined below, focusing
on uncoordinated solutions (solutions that can be implemented unilaterally) and
coordinated solutions (solutions that require collecive action). Both technical
and non-technical solutions are discussed.

\section{Uncoordinated solutions}
The first set of solutions that we can consider to this problem are
uncoordinated solutions---measures that require only individual action to
achieve individual benefit in terms of mitigating the effects of routing table
growth. Solutions in this class are easy to deploy because they do not require
the collective action that a protocol transition would entail. However, they
are also somewhat limited in that they must work within the existing technical,
business, and policy architecture of the Internet and its interdomain routing
system. In other words, these solutions cannot change the underlying scaling
properties or route aggregation incentives of our current system, and instead
use other tactics.

In this class of solutions there are several technical approaches that have
been proposed as discussed in the related work. Generally these involve
spreading the routing table across multiple routers (virtual aggregation)
\cite{Ballani:2009tg} at the cost of adding additional hops and packet
processing latency, or performing aggregation of the RIB inside the router. The
latter approach, which is similar to the aggregation performed by the CIDR
Report (though almost certainly uses a different criteria for the ``same
routing policy'' invariant, such as next-hop IP address
% also, ACLs will affect aggregability -- whooops
), is a solution preferred by network operators \cite{Zhao:2010fu}, and could
certainly improve the longevity of the more expensive FIB routing table slots
for individual networks that deploy this solution. Further, it is likely that
the operators affected by the size of the routing table growth would implement
a solution like this (providing their vendors supported it), allowing other
networks to continue to rely on them as well. The in-router aggregation
approach, if applied with similar results to the CIDR Report could reduce the
size of the FIB by approximately half\footnote{The ratio of prefixes in the
fully aggregated table to prefixes in the current routing table is
175224/355045 $\approx$ 0.49 as of 28 January 2011, using our implementation
and data sources.}. Table growth induced by IPv4 transfers and IPv6 would not
typically be aggregable, but this approach could make more room for these new
sources of prefixes by reducing the size of the routing table due to traffic
engineering, etc.

Also in this class are non-technical solutions. Unlike the technical solutions
which mitigate the effects of routing table growth, these are proposed to work
by creating incentives for announcing fewer or more aggregated routes. One
option in tthis space is to continue using the status quo CIDR Report. This is
included here because of the acknowledgement that the CIDR Report is not a CPR
governance institution on its own, and so without a strong institution using
the information it provides, it can at best inspire individual action to
persuade or induce others to reduce their impact on the routing table. Given
that the CIDR Report appears to be ineffective now, both from operators'
perspectives and from our analysis, it could possibly be improved based on the
observations in the previous section in hopes that it might better activate the
latent norms-based behaviors that seemed to make the CIDR Report effective in
the past. However, this does not appear to be an extremely promising option
given its recent history, operator attitudes towards the report, the problems
identified in Chapter \ref{chap:discussion}, and the fact that the technical
solution above would likely be more effective.

Bilateral economic solutions may also be possible. For example, a Tier-1
provider could begin to charge its peers and customers for the number of routes
they advertise. However, this would require all of the peers and customers to
acquiesce to this change instead of taking other action, such as switching
to a provider that does not charge for route announcements. The autonomous
nature and competitive forces of the Internet might thus make such bilateral
solutions untenable unless implemented by all Tier-1 providers simultaneously,
and so economic solutions of this nature are probably better considered in the
following section instead.

\section{Coordinated solutions}
Beyond individual technical solutions described previously, it appears that the
most promising solutions to managing the size and growth of the routing table
will require coordination and collective action to achieve. Whether technical
solutions that involve adoption of a new interdomain routing protocol, economic
mechanisms to settle the costs of propagating a route across the Internet, or
a CPR governance institution for interdomain routing, all of these approaches
will require the cooperation of most of the participants in the Internet's
routing system in order to be effective and realize benefit from the solution.

Regardless of the solution or approach, there are general challenges in
achieving collective action. First, as Olson\footnote{Rise and Decline of
Nations or Logic of Collective Action} and others point out, achieving
collective action can be difficult when benefits are not concentrated, or are
not selective, creating an incentive for opportunistic free riding. Also, while
not uncommon in other spheres of life, the notion that the Internet is no
longer composed of actors with homogeneous interests is relatively novel, as
identified by \cite{Clark:2005rt}. Instead of a cadre of similarly-minded
technologists, there are now a much more diverse and potentially conflicted set
of interests at play in shaping the architectural evolution of the Internet, as
well as achieving their objectives through the architecture. Even CPR
governance instituions are not guaranteed to form in otherwise beneficial
situations, as Ostrom \cite{Ostrom:1990fv} identifies, because there is a
constitutional collective action problem (a meta-problem) in agreeing to
develop an institution in the first place before the institution can then be
used to resolve operational collective action problems. Acknowledging that
collective action will be required, and may be difficult, we now sketch out
some potential approaches in this space.

\paragraph{A new interdomain routing architecture}

The most common approach taken thus far when approaching the limits of an
Internet protocol or architecture is to replace the protocol with a new, more
scalable version (e.g.  NCP to IP(v4), IPv4 to IPv6, EGP to BGP, etc.). The
adoption of new protocols is challenging; this challenge increases when a
protocol is not backwards compatible with its predecessor, and even more at the
network layer of the protocol stack, where the new protocol must be coordinated
along the entire end-to-end path between hosts, rather than just on the hosts
themselves. This dilemma often means that there is no benefit from or incentive
for incremental protocol adoption, thus making adoption difficult to justify
for rational, commercially-motivated actors. This, at least at a high
level, is plainly visible in the slow-moving adoption of IPv6 that has
theoretically been ongoing for the past 16 years \cite{rfc1883} and has only
recently started to gain traction with the depletion of unallocated IPv4
address blocks.

In this context, we can consider the adoption of a new interdomain routing
architecture and protocol

ILNP \cite{Atkinson:2010zr} may have killer apps, but ipv6 and routing table
growth are not them, at least until it becomes painful

\paragraph{Economic solutions}
- economic/market/contractual solutions

- maybe not "pricing" but just a mechansism where route advertisement is
  compensated through contracts like those discussed in the literature

\paragraph{Governance institutions}
- more structured CPR governance institution for the routing table

- who would lead this?
    - people who suffer -- Tier-1s/DFZ/etc?
    - perhaps these people could come together for a partial
      solution?
%- improve social/CPR governance?
%  - need to be able to answer questions about what is "fair"? ("know it when
%  you see it?") - would something like a routing table governance council
%  actually allow continued enjoyment of the dynamic aspects of internet
%  routing?
%  - need sanctions (see what else ostrom has to say about this)
%    - how do you actually do this? routing, but there is a problem of adoption
%    and market power and who has influence

% Sanctions are a challenge, and while there are some opportunities available
% within the routing system, the economic or operational ``damage'' that can be
% done is proportional to the content or customers you serve. Looking at other
% mechanisms for unilateral or multilateral sanctions within the Internet is a
% major opportunity for future work.

% This suggests that automated sanctions or pricing might be more effective.
%
% The challenge of agreeing to supply an institution that does anything more
% than provide information (i.e. sanctions) multilaterally is potentially
% problematic in looking beyond the CIDR Report.
%
% Interconnection is voluntary -- libertarian vs. collectivist thought

\section{Concluding thoughts}
The CIDR Report is a historic artifact that has ceased to be useful on today's
internet -- it was designed for a different purpose and focused on different
behavior than might otherwise be desired. However, it seems it was indeed
effective, and this is a useful lesson in considering solutions to problems and
designing new institutions for the Internet.

\begin{itemize}

\item{technical solutions in the short term -- driven by short-term pain or
forward thinking}

\item{protocol replacement in the long term -- likely driven by other benefits
or pain (as with CIDR deployment), though more difficult to coordinate}

\item{coordination will be difficult regardless -- CIDR is relatively
miraculous by comparison and so the IETF/IAB, router vendors, and operator
communities will need to think about how to execute our next transition
effectively and with the proper incentives, hopefully learning lessons from teh
difficulties of IPv6.}

\item{governance institutions would be useful in solving a number of the
problems of the Internet, and perhaps more efficiently, but may be difficult to
scale and coordinate Internet wide, and possibly also ideologically difficult}

\item{CPR institutions still hold promise -- a call to action for these people
to consider that or allow it to function within}

\item{RIRs serve a good ``nucleation point'' as a forum where providers can
organize into regionally-scaled CPRs}

\end{itemize}

\section{Future work}
This work on the CIDR Report and social forces at play in the Internet
operations community have exposed a rich areas for further study in
understanding the effects of the CIDR Report and the nebulous underlying
governance mechansism that enabled these effects. In terms of routing table
growth more broadly and Internet collective action and governance institutions
more broadly still, the use of governance institutions as an alternative to
other proposed solutions for collective problems facing the Internet could use
more study and exploration about potential opportunities therein.

With regard to this work, there are several methodological improvments that
should appear in future iterations of this work that would greatly improve the
quality of the analytic process and hopefully reduce a number of the
confounding factors identified previously. First, measurement of the
pre-treatment behavior of ASes that would eventually appear on the CIDR Report
would allay concerns that some behavior other than appearance on the CIDR
Report was causing changes in aggregation behavior. Perhaps more importantly,
construction of a more representative control group (such as the ASes who are
just shy of appearing on the CIDR Report) would afford greater confidence that
the observed behavior change following a CIDR Report appearance was not due to
some of the biases mentioned in Chapter \ref{chap:analysis}. Finally, it would
better to utilize a statistical test, such as a Kolmogorov-Smirnov test, to
determine whether the treatment and control behaviors actually differ
significantly.
% Finally, performing a sensitivity analysis on some of the assumptions and
% design decisions built into our analysis would expose the

Beyond these methdological improvements, there are also a number of further
analyses and new directions to explore regarding the CIDR Report. Improving our
analytical tools to allow more fine-grained analysis of CIDR Report response
over time and amongst classes of ASes (i.e. ISPs vs. others) could expose other
behaviors not observed in this analysis. Our observation about the sensitivity
of the CIDR Report to variations in prefixes received from peers at the vantage
point may also merit further study to estimate the degree to which outlier
prefixes have contributed to inflated rankings on the CIDR Report.  Finally,
given our observations about attitudes towards deaggregation and potential
causes of the diminished response to the CIDR Report, it would be interesting
to explore alterante constructions of a report, such as a report ranked by
deaggregation factor or a report that highlighted recent increases in
deaggregation.

%- consult with the Internet operations community further

Finally, there are potentially interesting opportunities in considering the
role that CPR governance institutions could play in helping to solve some of
the collective action problems facing the Interent and Internet operators. This
is not so much an extension of this work as suggestions for future directions
based on this work. First, as Ostrom observed in her list of empirical design
patterns for long-standing CPR governance institutions, there is a need for
sanctions to ``back up'' norms and punish opportunism---something which the
CIDR Report lacked. However, in a globally distributed Internet community that
transcends borders and laws, many of the traditional economic or legal tools
that might normally be used to implement sanctions are ineffective or
unavailable. It would thus be interesting to consider technical and social
mechanisms that could possibly be used to implement sanctions or otherwise
disincentivize opportunistic behavior that is against the collective interests
of Internet operators and users. The development of such mechanisms would
likely require a degree of collective action itself, which leads more generally
to the need for appropriate forums for participants to come together to propose
and commit to collective action. While forums such as network operator meetings
and the IETF exist today, these may not be the correct forums, both because of
their existing focus and history, and because of participants fear of the
appearance of collusion and the resulting consequences from antitrust law.
Such work could be a valuable contribution to thinking about improvements to
and altogether new Internet architetures.

% TODO in terms of mechanism and complexity, are CPR institutions + sanctions
% more efficient than markets for things that aren't necessary truly valuable
% (like routing table slots?)



% Economic arguments and palatability in considering replacement protocols in
% the IETF

% Could a "cartel" of DFZ/Tier-1 providers start this because they have the
% disciplinary ability? We can look to the minimum prefix length requirements in
% the IPv4 days (more research required) and in IPv6 with the /32 minimum with
% Verizon, and how this got washed away by an interest in serving the market --
% apparently the incentive to curtail routing table growth just isn't there...?

% While the norms and values embedded in the Internet community may have been
% helpful in past, it's not clear that these are or will continue to hold (cite
% Tussle in Cyberspace) and so we probably shouldn't expect the CIDR report to
% work in the same way, especially with expansion into other regions where the
% culture/Internet ethos isn't the same, or where people aren't aware of the
% history, or loosely/not-at-all connected to the community that does still
% exist.

% As ostrom points out, there's a meta-problem of supplying instutions and
% collective action in bringing about the instiutions one wants to exist to
% solve the first-order collective action problem in the first place.

% Citadinni et al. also point out that the bad guys have stayed the same
% generally, since 2001 -- again, not helpful
%
% This suggests that automated sanctions or pricing might be more effective.
%
% The challenge of agreeing to supply an institution that does anything more than
% provide information (i.e. sanctions) multilaterally is potentially problematic
% in looking beyond the CIDR Report.
